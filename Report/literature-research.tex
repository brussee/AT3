\documentclass[11pt]{article}

\usepackage{graphicx}
\usepackage{epstopdf}
\usepackage{longtable}
\usepackage{caption}
\usepackage{pdflscape}
\usepackage{bbding}
\usepackage{pifont}
\usepackage{wasysym}
\usepackage{amssymb}
\usepackage[bookmarks]{hyperref}

\title{Action plan}
\author{Rolf Jagerman, Laurens Versluis and Martijn de Vos}
\date{\today}

\begin{document}

\maketitle

\pagebreak

\section{Introduction (Martijn)}
Since the introduction of the internet in 1991 by CERN, we can't imagine a life without it. We make extensive use of applications like Facebook, Twitter and Netflix, applications that have only recently emerged. We're using the internet to share moments of our life, to run our business or for our entertainment by playing online games or watching videos. The innovation continues at a high rate and we're using the internet more and more.\\\\
We're not only connected with a desktop computer or laptop anymore: mobile devices are raising in popularity. Mobile devices like smartphones and tablets allows us to be always connected with each other. Popular mobile applications like WhatsApp and Instagram often have millions of users.\\\\
The increasing use of the internet also had its downsides. Cybercrime is a recent and growing problem: we're not always safe when we're browsing the worldwide web. Scamming and hacking are serious problems of the 21th century. In some countries like China and North Korea, the internet is censored and inhabitants are prohibited from visiting websites that are forbidden by the government. Anonymous communication could be a solution for these countries. A popular network for anonymous communication is Tor, however, as we will see Tor has many problems and does not scale very well. Another step in the right direction is Tribler, an anonymous peer-to-peer network created by the TU Delft. The rise of the internet also brings discussion about net neutrality, which is a highly debated issue in many countries.\\\\
In this research, we will first describe the Tor network in section II and look into its advantages and disadvantages. After that, the focus will move to Tribler and we will discuss the Tribler system in section III. We will delve into the various components Tribler contains of. In section IV, the focus will move to the mobile platform Android. We will look into the Android platform and discuss the advantages of the mobile operating system and what it could mean for Tor and Tribler. In section V, we will discuss a library that allows to run Python code on an Android device: Python for Android. One of the contributers to Python for Android is The Global Square. We will explain what The Global Square exactly is and what their goals are in section VI. Finally, we will discuss how we will be using the discussed software for our projects and we will elaborate what our project exactly is.

\section{Tor (Rolf)}
Hier komt wat over Tor

\section{Tribler}
Hier komt wat over Tribler

\subsection{What is Tribler? (Rolf)}

\subsection{M2Crypto (Martijn)}
Security is a big issue in the world of peer-to-peer networks. Not only do we want anonymous downloads, we also want confidentiality and integrity of our data. Confidentiality means that unauthorized parties can't see the exact content of the information. This could be achieved by encrypting the data. Integrity of the data means that the data is protected from being modified by other parties. Integrity can be achieved taking the hash of the data you receive and comparing it by taking the hash of the original message. If the hashes are not equal, the message has been modified. Confidentiality and integrity are important. When we're sending confidential data over the internet, it would not be beneficial if everyone has access to the message and can read what is being send. Integrity can play a role when transferring money to another bank account: we do not want an adversary to temper with the amount of money that is being transferred.\\\\
Some popular open source frameworks exists for these cryptographic tasks. A popular project is OpenSSL (http://www.openssl.org). OpenSSL implements the popular SSL and TLS protocols. These are cryptographic protocols that provides security when communicating over the internet. Besides that, OpenSSL provides libraries for various encryption and decryption protocols such as DES, RSA and RC4. OpenSSL also supports key exchange protocols such as Diffie-Hellman.\\\\
OpenSSL is written in C. To use the OpenSSL libraries in Python, one could use pyopenssl (https://github.com/pyca/pyopenssl), a interface for OpenSSL or M2Crypto (https://github.com/martinpaljak/M2Crypto), an OpenSSL wrapper. Tribler makes use of the M2Crypto library. The old homepage of the M2Crypto project (http://chandlerproject.org/Projects/MeTooCrypto) explains what M2Crypto is:\\\\
\emph{M2Crypto is the most complete Python wrapper for OpenSSL featuring RSA, DSA, DH, HMACs, message digests, symmetric ciphers (including AES); SSL functionality to implement clients and servers; HTTPS extensions to Python's httplib, urllib, and xmlrpclib; unforgeable HMAC'ing AuthCookies for web session management; FTP/TLS client and server; S/MIME; ZServerSSL: A HTTPS server for Zope and ZSmime: An S/MIME messenger for Zope. M2Crypto can also be used to provide SSL for Twisted.}\\\\

\subsection{Tribler Mobile}

\subsection{Dispersy (Laurens)}
Dispersy [CITE NAAR DISPERY PAPER] is a fully decentralized system for data bundle synchronization used by Tribler. The system is designed in such a way that it is capable of running in a challenged network environment. Such an environment is often characterized by:
\begin{itemize}
\item Nodes randomly joining and leaving
\item Delays in the network
\item Nodes having different networking speeds (Edge, 3G, WiFi).
\item Nodes often being behind routers with Natwork Address Translating (NAT) firewalls.
\end{itemize}

All communication done by Dispersy uses UDP, because up to 64\% of the Internet is behind a NAT, they can use UDP firewall-NAT puncturing mechanisms.\\

In Dispersy, each node has a candidate list. A candidate list is a list of active connections within the node's overlay. A Dispersy node synchronizes in five steps:

\begin{enumerate}
\item First it selects a note from its candidate list.
\item It then selects a range of bundles to synchronize.
\item The node creates a Bloomfilter by hashing the selected bundles.
\item Then the node sends the created Bloomfilter to the selected node.
\item Finally, it pauses for a fixed interval to go back to step 1.
\end{enumerate}

The candidate list is divided into three sections: trusted nodes, nodes that have been successfully contacted in the past and nodes that have been connected in the past either trough an introduction-request or nodes that have been introduced.\\

A Bloomfilter use a hash area consisting of N bits, initially all set to 0. For each item that needs to be stored in the Bloomfilter, K distinct addresses are generated using a hash of the item. The bits addressed in the Bloomfilter are then set to 1. To check if an item is part of a Bloomfilter, one only has to generate the hash of that item and check if the addresses that are generated by the hash are 1 in the Bloomfilter.\\

After benchmarking Dispery against Cassandra (the database system used by Facebook), they came to the conclusion that Dispersy performs better than Cassandra. By using Bloomfilters, Dispersy can scale to over 100,000 bundles to synchronize.


\subsection{Anonymous tunnels (Martijn)}

\section{Motivation}
Worldwide the amount of mobile devices is growing fast. In 2013, there were 6.8 billion mobile subscriptions worldwide (http://www.itu.int/en/ITU-D/Statistics/Pages/stat/default.aspx). This number is expected to exceed the world population in 2014 with 7.3 billion subscriptions. Android had a share of 81\% in Q3 of devices shipped (http://www.forbes.com/sites/tonybradley/2013/11/15/android-dominates-market-share-but-apple-makes-all-the-money/).

This shows that there is a lot of potential gain on the Android market. Android is open source and can be modified, on top of that is the Google Play not subject to a review before you can release your application, which allows for an easier deployment. Since a library called Python for Android already exists, it's also a more natural choice to develop it for Android since as stated earlier, the Tor-tunnel functionality is written in Python.

Besides the arguments stated above, a mobile application is an excellent way to attract more new users to Tribler. With the billions of mobile devices out there, this could be a interesting new area to explore.

\section{Python for Android (Rolf)}
Hier komt een stukje over wat Python for Android precies is, wat het doet en hoe we het kunnen gebruiken.

\section{The Global Square (Martijn)}

\subsection{What is The Global Square}

\subsection{Their contribution to Tribler}
Hier wat over de libraries die ze hebben gecompileerd voor Android

\subsection{Our project (Martijn)}
Hier wat over ons project

\end{document}
