\chapter{Conclusion}
	We come to the conclusion that the current generation smartphones are not ready yet for anontunnels, without gmpy. The project itself was successful; we managed to cross-compile all required libraries. The anontunnels are running and downloading when the hop count is low. Downloading over multiple circuits is possible and relaying of data works too. By creating a pull request, we hope to provide the Python for Android with some useful recipes we created during our bachelor project, contributing to the open source community. 
	All of our work, including this report, is open source and can be extended. Jaap will merge our project with the other bachelor project group TSAP, creating an application that allows anonymous en encrypted downloading of movies, with a search and playback option.
	
	
	Once smartphones running on ARMv8 become available or if gmpy is integrated in the project, we foresee that this application will be able to run with 3 hops and multiple circuits. Once that is possible, Tribler takes a censorship-free internet to a new level.
	
	\section{Recommendations}
		If one wishes to extend our work, we recommend to try and get gmpy to compile by creating a recipe for it. This might speed up the cryptographic part, allowing for multiple hop with encryption downloading. This recipe can also be pull requested to the Python for Android GitHub repository.
		
		We would also recommend working closely together with the Tribler team, as the application integrates the code of the Tribler main code base. The Tribler team often can help with question or advise you on best practice regarding their code.
		
		Finally, we recommend to keep up to date with dependencies whenever you can. The Tribler code base changes regularly, so it is good practice to keep up-to-date. Other libraries such as OpenSSL have had some fixes regarding the famous Heartbleed bug. Applying patches like these might be critical in terms of functionality or security.