\chapter{Plan of Action}
\label{cpt:planofaction}

This plan of action is part of the bachelor project on anonymous video streaming on tablets. In this project we attempt to get Tribler, a peer-to-peer file sharing application working anonymously on Android using an already implemented Tor-like protocol.

In this appendix we outline the details of the assignment. We will describe the approach we will use and how we will structure this project. Finally, we will describe how we maintain the quality of our work.

\section{Assignment}
In this section we will describe our assignment, client, contacts, the problem we will address and sketch out what product we will eventually deliver. This section will also contain some of the critical requirements that we will have to meet along with the risk involved.

\subsection{Assignment}
Our assignment is to implement a new feature of Tribler into a mobile android application. This new feature allows the creation and usage of so called anontunnels. These tunnels allow anonymous download within a peer-to-peer network between devices, in our case Android smartphones.
As these tunnels run on Python code, we will have to be able to run Python on an Android smartphone, along with all the libraries it depends on.

\subsection{The client}
Our client is Dr. Ir. Johan Pouwelse, the head of the Tribler group and Assistant Professor at the Parallel and Distributed Systems Group of the Faculty of EEMCS, Delft University of Technology. Pouwelse has measured and researched peer-to-peer networks for years and has been working on Tribler for nine years.

\subsection{Contacts}

\emph{The Client:}\\\\
\textbf{Dr. Ir. Johan Pouwelse}\\
J.A.Pouwelse@tudelft.nl\\
+31 (0)15 27 82539\\
Room: HB 07.290\\
Mekelweg 4\\
2628 CD Delft\\\\

\noindent\emph{TU Delft coach:}\\\\
\textbf{Ir. Egbert Bouman}\\
E.Bouman@tudelft.nl\\ 
Room: HB 07.290\\
Mekelweg 4\\
2628 CD Delft\\\\

\noindent\emph{Bachelor Project Coordinator:}\\\\
\textbf{Dr. Martha A. Larson}\\
M.A.Larson@tudelft.nl\\
+31 (0)15 27 87357\\
Room: HB 11.040\\
Mekelweg 4\\
2628 CD Delft\\

\subsection{The final product}
\label{ssec:final-product}
At the end of the bachelor project, we will deliver an Android application that allows users to find and download content anonymously. This application makes use of the code from the Tribler project, which is written in Python. Therefore, we will make sure that the Android application will be able to run Python code.\\
The downloads that run through the anontunnels will be anonymous, just like the current Tor protocol.

\subsection{Requirements and risks}
The final product will offer the features that are described in Subsection \ref{ssec:final-product} and should be considered a prototype. The prototype will offer anonymous downloading using the anontunnels mentioned earlier. The development of the prototype will be targeted to the Android platform.\\\\
The following requirements are set:

\begin{itemize}
\item Weekly Scrum evaluations. At the end of each Scrum iteration, we evaluate what we have done and set the target for next week. This keeps the deadlines SMART\footnote{\href{http://www.techrepublic.com/article/use-smart-goals-to-launch-management-by-objectives-plan/}{www.techrepublic.com/article/use-smart-goals-to-launch-management-by-objectives-plan/}} and manageable.
\item Weekly meeting with the client. Every two weeks we will implement a feature, but we will show and discuss our progress each week with the supervisor.
\item The members of the team will complete a prototype at the end of this project and will also demonstrate this during a 30 minute presentation given in the last week of Q4.
\end{itemize}

The risks involved with this project include:
\begin{itemize}
\item Run Python code on an Android application. As the Tor-tunnel functionality is written in Python code, we need to be able to run Python code on an Android device as well. A library exist where you can write Python for Android, but we will still face a challenge when we will try to combine other libraries.
\item As we are dependent on third party code, we might lose time to understand or read certain parts of code or documentation as well as link pieces of code that belong to different parties together.
\end{itemize}



\section{Approach}
In this section, we will describe the approach we are taking for this project. First, we will discuss the methodology we are using (Scrum). After that, we will discuss the MoSCoW technique we are using to classify requirements. Afterwards, we will give an overview of the tools we will be using during this project. Finally, we will give our planning and milestones.

\subsection{Scrum methodology}
For the project, we will be using the Scrum methodology. We have used Scrum in various projects already during our studies and it has proven to be a very efficient way of working. In this section, we will discuss how we plan to use Scrum and what our Scrum iterations will look like. First we will look at the different roles involved in the Scrum process. After that, we will describe how the Scrum process is organized.

\subsubsection{Scrum roles}
There are three primary roles involved in Scrum:
\begin{itemize}
\item Product owner: the product owner represents the stakeholders and is the voice of the customer. He is responsible for the success of the product. Johan Pouwelse is the product owner.
\item Development team: the development team consists of Rolf, Laurens and Martijn. We are responsible for delivering the final product to the product owner.
\item Scrum master: he guides the team by assuring the right choices are being made. He is responsible for arranging the meetings. Rolf is our Scrum master.
\end{itemize}

\subsubsection{The Scrum process}
There are several steps involved in the Scrum process. First, we will create a product backlog. This is a list with the functional demands the product owner has and it contains the items we still have to do. In total, we have 5 sprints. At the end of each sprint, we will deliver a part of the final product. The duration of each sprint is two weeks. At the beginning of each sprint, we will create a sprint backlog. This backlog describes the functional demands, divided in each subtask for this sprint.\\\\
Each morning, we will start the day with the daily Scrum. This is a short meeting where every member of the development team answers the following question:
\begin{itemize}
\item What have you done yesterday?
\item What are you going to do today?
\item Are there any problems you ran into?
\end{itemize}
At the beginning of each sprint, we start with a meeting. This meeting is attended by all members of the team and the supervisor. The purpose of this meeting is to evaluate the last sprint and decide on the tasks that have to be done during the next sprint.

\subsection{MoSCoW}
MoSCoW is a technique that can be used to place importance on the delivery of each requirement. During each sprint, we will classify the features in one category. The MoSCoW model has the following categories:
\begin{itemize}
\item Must have: the requirement must be part of the final product to be considered a success.
\item Should have: the requirement has a high priority and should be in the final product.
\item Could have: the requirement is desirable but not necessary.
\item Would have: the requirement is not implemented in a given release but is considered as a requirement in the future.
\end{itemize}
At the start of each sprint, we evaluate the goals we want to achieve that sprint. After that, we classify each goal into one of the categories above. Since we do not have everything clear at the start of the project, it could happen that we prioritize the goals differently during each sprint.

\subsection{Tools}
For this project, we will use various tools, both hardware and software. First of all, we will be using our own laptops for the development of the software. We will develop our software on the Ubuntu platform. We are also using Android phones that we can rent from the Tribler team.\\\\
If we look at the software, we will make use of the Android SDK and NDK. The Android SDK will allow us to build .apk files. The NDK allows to implement parts of an Android application in C or C++. To be able to run Python code on an Android device, we will use the Python for Android library.

\subsection{Planning}
Our planning can be found on GitHub. As for now, we have four milestones:
\begin{itemize}
\item 02-05-14: the literature research should be done and a report about the read literature should have been written.
\item 09-05-14: we should be able to compile the TGS for Android project and run it on an Android device.
\item 30-05-14: we should be be able to send a packet between two Android devices over anonymous tunnels.
\item 13-06-14: a GUI for testing purposes should be designed and created.
\item 27-06-14: the end presentation of our project.
\end{itemize}

\section{Project structure}

In this section the administrative aspects of the project are described.

\subsection{Members}
The members of the project are Rolf Jagerman, Laurens Versluis and Martijn de Vos. All members will work 40 hours per week to ensure the mandatory 15 EC per student are utilized. The division of labor is evenly distributed across all activities (analysis, documentation, implementation, etc.).\\

\noindent\textbf{Contact Information}\\
Rolf Jagerman - R.M.Jagerman@student.tudelft.nl\\
Laurens Versluis - L.F.D.Versluis@student.tudelft.nl\\
Martijn de Vos - M.A.Devos@student.tudelft.nl

\subsection{Reporting}
Weekly meetings with the client will be held in person. All documentation of the project will be written in {\LaTeX} and will be provided as a PDF. The project material, including source code and documentation, will be available throughout the project on the AT3 GitHub repository\footnote{\href{http://www.github.com/rjagerman/AT3/}{www.github.com/rjagerman/AT3}}.

\section{Quality assurance}

To assure a good quality of the delivered product, several agreements are made about which methods should be used. In particular we look at testing, code review and version control.

\subsection{Testing}
All written software will be tested using Python unit tests. Additionally, test code coverage is provided to ensure a majority of the code has been thoroughly tested.

\subsection{Code review}
All written code will be reviewed by at least one team member before pull requests are accepted. This will ensure the code is comprehensible, working and correct. Additionally our code will undergo a complete source review by SIG (Software Improvement Group).

\subsection{Version control}
To maintain a good overview and history of the code we write, we will use a version control system. All our code will be stored on GitHub and therefore use the Git version control system.
