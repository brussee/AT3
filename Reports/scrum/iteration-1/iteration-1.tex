% under the arm architecture -> under het juiste woord?
% in section 6.2 duidelijker het arm build proces beschrijven
% packages duidelijker beschrijven hoe ze gebruikt worden en wat ze inhouden
% porting the anonymous tunnels to Android korter of weg
% duidelijker het verschil aangeven tussen GNU en ARM compiler
% creep in anders verwoorden
% letten op de tijden in 6.4.2
% link bij automake / Unix
% graphical user interface definieren
% conclusion - we hebben niet alle goals bereikt
% evaluatie in plaats van conclusie aan het einde van elke sprint?
\chapter{Scrum iteration 1 - creating a basic application}
\label{cpt:iteration1}
	In this chapter we describe our first sprint that lasted two weeks.

	\section{Goals}
		This sprint, we had the following goals:
	
		\begin{enumerate}
			\item Get Python code working on Android. Evaluate and set up Python for Android (must have feature because this is the basis of our whole project).
			\item Get all packages that are needed to run the anonymous tunnels working on Android (should have, we should have most packages working but if there are some package that do not work yet, it is not that bad).
			\item Implement a basic GUI to test with. This GUI should be created with Kivy (could have, we can use logcat if we do not have a GUI available).
		\end{enumerate}
	
	\section{Python for Android}
		The Python for Android framework allows developers to add existing Python packages by creating recipes (for more information about recipes, see Section \ref{sec:pythonforandroid}). As most of the packages were not available or were custom made, we had to build a lot of these recipes ourselves.
	
		Python for Android allows to build these packages for the ARM architecture so they will not compile for the desktop architecture. This desktop architecture is not suitable for Android development. By simply using the \emph{push arm} and \emph{pop arm} commands in the recipe files, compilation for the ARM architecture can be triggered.
	
		Below are the packages described we use in our application and in most cases had to create a recipe for. We describe the functionality of the package in our application and which dependencies this package has.
	
		\begin{itemize}
		
			\item Kivy\\
			We use Kivy as framework for creating the GUI. Kivy is an open source software library for creating Natural User Interface (NUI) applications. It is easy to use and cross-platform, allowing users to create a GUI on their PC and then integrate it in their products (we integrate it in our Android application). As Kivy is integrated in Python for Android, it is a natural choice to use it for create the GUI. Kivy is dependent on Python, as it is a Python package. 
		
			\item OpenSSL\\
			OpenSSL is the world's most famous open source cryptography toolkit, also available for Python. As our application makes use of PyCrypto and M2Crypto which are both dependent on openSSL, we have to include it in our app.
		
			\item M2Crypto\\
			Dispersy and Tribler are dependent on M2Crypto as it has some security features which M2Crypto implements such as elliptic curves cryptography. As the anonymous tunnels, our core function of the application, are dependent on both Dispersy and Tribler we also need M2Crypto. M2Crypto itself is dependent on Python and OpenSSL as it is a Python package using the OpenSSL implementation.
		
			\item PyCrypto\\
			PyCrypto is a Python library which implements certain cryptography functions used by the tribler\_core\_minimal package. PyCrypto itself depends on functionality of the OpenSSL package.
		
			\item Boost\footnote{http://www.boost.org/}\\
			The Libtorrent library is used to download torrent files with the Tor protocol. To compile Libtorrent, the Boost package is required. Boost is a huge C++ library which adds code for multithreading, regex, math and asynchronous operations. We have to compile the Boost library with Python bindings enabled so we can invoke the library from Python code.
		
			\item netifaces\\
			netifaces allows to easily get the address(es) of the machine's network interfaces from Python. It is therefore dependent on Python and used by the Dispersy package.
		
			\item Zope\footnote{http://www.zope.org/}\\
			Zope is a open source web framework for object-oriented web application servers. The Twisted package makes use of this framework for asynchronous networking.
		
			\item Twisted\footnote{https://twistedmatrix.com/trac/}\\
			Twisted is an extensible framework for asynchronous networking written in Python. The framework has special focus on event-based network programming and multi-protocol integration. It has dependencies on the Zope framework. In the Tribler software, Twisted is used for callbacks of network events.
		
			\item anontunnels\\
			This package is the core of our application and contains the code needed for the anonymous tunnels. This package actually contains another package: the support for the Socks5 proxies. The files for this package come from pull request 525 on the Tribler GitHub.
			
			We made some minor changes to this code: the Main.py file has been removed from the package and the class definition of AnonTunnel has been moved to its own file (androidinterface.py). We import this file in our main.py so we can use the AnonTunnel class. We also adjusted the master key in community.py so we have our own community to test communication between devices.
			
			\item Tribler\\
			This package has been created from the Tribler main branch\footnote{https://github.com/Tribler/tribler}. Some files are modified to make the library run without problems for Android.
			
			\item Dispersy\\
			Since the code of the anonymous tunnels are using Dispersy for node discovery and data synchronization, we have created a Python package with all the code thats needed for Dispersy. This package does not depend on other custom packages we made and can be used standalone.\\
			The files we have bundled are from pull request 525 on the Tribler GitHub. We did not use the files from the official Dispersy GitHub because this build was missing some classes we needed (for example, the decorator.py). Besides that, some changes in the pull request have been made to the Dispersy core to add support for the anonymous tunnels.
		
		\end{itemize}
	
	\section{Porting the anonymous tunnels to Android}
		One of our first challenges was to port the code that setup the anonymous communication to the Android device, using Python for Android. We started by inspecting the current code from pull request 525 and dived into the dependencies this code has with the Tribler core and Dispersy. We decided to create three packages: one package with the Dispersy code, one package with the files we needed from the Tribler core and one package containing the anonymous tunnels code.
		
		After we created these packages, we had to find out which other packages we needed to run everything. We did this, by putting many imports in our main.py file which is being run on the device. Each test run, we looked at the error which was being thrown. Most of the time, this was an import error, indicating we missed some code from other packages. We used various packages that The Global Square has ported such as M2Crypto and netifaces.
		
		Import errors were not the only issues we ran into: we had some problems with the netifaces package. Since this package is copied into the final APK file as a Python egg\footnote{http://mrtopf.de/blog/en/a-small-introduction-to-python-eggs/}, it will be extracted on the device.
		
		We also found out that some files were missing and not copied into the APK. The curves.ec file was missing and is needed by the cryptography classes found in the Tribler core. We also needed the configuration file of the logger, logger.conf. To make sure these files are part of the application, we copy them into the final application in our build script.
		
	\section{Attempt to compile Libtorrent for Android}
		To get libtorrent working on Android there are several big obstacles:
		\begin{itemize}
			\item Compiling Boost for the ARM architecture
			\item Compiling libtorrent for the ARM architecture
			\item Compiling libtorrent Python bindings for the ARM architecture
		\end{itemize}
		
		The official documentation of libtorrent states multiple ways to build libtorrent. The first is using Boost's build system \emph{Jam}. The second is using \emph{automake}. However, before we can compile the libtorrent source code with either of these methods, several modifications had to be applied.
		
		\subsection{Source code modifications}
				
		We found several modifications to the source code to be necessary to compile libtorrent. A special thanks to Steeve (\url{https://github.com/steeve/libtorrent/}) for these modifications:
		
		\begin{itemize}
		\item \texttt{INT64\_MAX} is not defined for Android, so we have to specifically define it.
		\item Multiple environments are defined in \texttt{include/libtorrent/config.hpp}. We add an environment for \texttt{ANDROID}, which sets the following options:
			\begin{itemize}
				\item \texttt{FALLOCATE} is disabled
				\item \texttt{ICONV} is disabled
				\item \texttt{IFADDRS} is disabled
				\item \texttt{MEMALIGN} is enabled
			\end{itemize}
		\item Instead of \texttt{<sys/statvfs.h>} we include \texttt{<sys/vfs.h>}, and redefine \texttt{statvfs} and \texttt{fstatvfs}. This is necessary because the Android's libraries only have \texttt{sys/vfs.h} and not \texttt{sys/statvfs}.
		\item Finally, we add an include for \texttt{<sys/syscall.h>} and redefine \texttt{lseek} to \texttt{lseek64}.
		\end{itemize}
		
		\subsection{Boost Jam}
		Boost jam\footnote{http://www.boost.org/boost-build2/doc/html/bbv2/jam.html} is a build environment created specifically for boost. However, it can be used to build other software. It is often included as a build option for software that is dependent on boost, such as libtorrent.
		
		Running boost jam is pretty straight forward. After executing \texttt{bootstrap.sh} we can run \texttt{b2} or \texttt{bjam} to compile Boost itself. With this command we specify the architecture which is described in more detail in the \texttt{user-config.jam} file. The resulting .so or .a files are compatible with the Android device.
		
		Compiling libtorrent using this method is more advanced. We again specify a \texttt{user-config.jam} with appropriate settings. However, the build process fails during compilation. The source for libtorrent will have to be modified in several places, because the ARM compiler and libraries differ from the normal GNU\footnote{http://gcc.gnu.org} compiler. After modifying the source code, the build still fails. Linker errors are occurring when you try to compile the Python bindings. The standard Unix\footnote{http://www.unix.org} libraries \texttt{pthread} and \texttt{util} do not have to be linked on Android, yet the boost jam environment forces these options for the Python bindings compilation. Due to the complexity of boost jam build environment, we decided to try using automake.
		
		\subsection{Automake}
		Automake is a standard set of tools for Unix-based systems that makes it easy to configure and compile software on a wide variety of systems. The tools are designed such that it is possible to configure the build process using a simple script.
		
		Following libtorrent's official documentation\footnote{http://libtorrent.org/manual.html} we first run \texttt{bootstrap.sh}. Now, in order to configure and compile we will have to set up an environment in which it will use the Android GNU ARM compiler. To do this, we set the following environment variables:
		
		\begin{lstlisting}
export SYSROOT=$ANDROIDNDK/platforms/android-14/arch-arm
export PATH=/usr/local/gcc-4.8.0-arm-linux-androideabi/bin:$PATH
export CC=arm-linux-androideabi-gcc
export CXX=arm-linux-androideabi-g++
export CROSSHOST=arm-linux-androideabi
export CROSSHOME=/usr/local/gcc-4.8.0-arm-linux-androideabi
		\end{lstlisting}
		
		Note that we have set up a custom NDK toolchain under /usr/local/gcc-4.8.0-arm-linux-androideabi. 
		
		Compiling libtorrent with this setup works, but the Python bindings still give linker errors. These are the same as errors Boost jam is using. The linker tries to link \texttt{pthread} and \texttt{util}, which are not required on Android.
		
		We will move several items involving libtorrent to the next scrum iterations:
		\begin{itemize}
		\item Compiling Python bindings for libtorrent
		\item Creating a proof-of-concept application to test if libtorrent works natively and with Python bindings.
		\end{itemize}
	
	\section{Creating a GUI with Kivy}
		The first version of the anonymous tunnels used the standard output for printing information about what is going on. This required the phone to be connected to a computer so we can examine the log with the adb logcat tool. That is why we decided to create a graphical user interface for our application. The purpose of this application is to provide a button to start the tunneling and a log to see on the screen what the status of the application is.
	
		Creating a GUI was a small step for us: we already included the Kivy package in our Python for Android distribution. Since Kivy is a GUI framework for desktop (but has been ported to Python for Android), we first tried to create a desktop interface. Creating interfaces in Kivy is quite straightforward and is related to creating user interfaces in Android: you specify your layout elements in Kivy files which have the kv extension. In the main Python file, you load this interface file and you can access properties of the UI elements.
	
	\section{Sprint evaluation}
		In this sprint, we have managed to complete almost all the goals -- with the exception of one creating one package package -- set for this sprint, and could even start with some tasks we had set for the next sprint.
		
		We have talked with Jaap, a member of the Tribler team. He told us that in his 20 weeks of work, he never managed to get the newest libtorrent to compile for Android.
		Currently he runs an old and modified libtorrent version. Since we managed to get in contact with Steeve, who got the latest version of libtorrent working with Go bindings on Android, we remain hopeful. He gave us advise on compiling libtorrent on Android, possibly with Python bindings. We have set the libtorrent package as a separate goal for the next sprint.
