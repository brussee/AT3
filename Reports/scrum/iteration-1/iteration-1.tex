\chapter{Scrum iteration 1: creating a basic application}
\label{cpt:iteration1}
	In this chapter we describe our first sprint that lasted two weeks.

	\section{Goals}
		During this sprint, we had the following goals:
	
		\begin{enumerate}
			\item Get Python code working on Android. We must evaluate and set up Python for Android (must have feature because this is the foundation of our whole project).
			\item Get all packages that are needed to run the anonymous tunnels working on Android (should have, we should have most packages working but if there are some packages that do not work yet, it is manageable).
			\item Implement a basic GUI to test with. This GUI should be created with Kivy (could have, we can use logcat -- the Android logging tool -- if we do not have a GUI available).
		\end{enumerate}
	
	\section{Python for Android}
		The Python for Android framework allows developers to add existing Python packages by creating recipes (for more information about recipes, see Section \ref{sec:pythonforandroid}). As most of the packages were not available, we had to build a lot of these recipes ourselves.
	
		Python for Android builds these packages for the ARM architecture. By using the \texttt{push arm} and \texttt{pop arm} commands in the recipe files, compilation for the ARM architecture can be triggered.
	
		Below are the packages described we use in our application and in most cases had to create a recipe for. We describe the functionality of each package in our application and which dependencies it has.
	
		\begin{itemize}
		
			\item Kivy\\
			We use the Kivy framework for creating the Graphical User Interface (GUI). Kivy is an open source software library for creating GUI applications. It is easy to use and cross-platform, allowing users to create a GUI on their PC and then integrate it in their products (we integrate it in our Android application). As Kivy uses Python for Android, it is a natural choice to use it. Kivy is dependent on Python, as it is a Python package. 
		
			\item OpenSSL\\
			OpenSSL is the world's most well-known open source cryptography toolkit, also available for Python. As our application makes use of PyCrypto and M2Crypto which are both dependent on openSSL, we have to include it in our app. More information about OpenSSL can be found in Subsection \ref{sec:security}.
		
			\item M2Crypto\\
			Dispersy and Tribler are dependent on M2Crypto as it has some security features which M2Crypto implements such as elliptic curves cryptography\footnote{\href{http://www.stanford.edu/class/cs259c/syllabus.html}{www.stanford.edu/class/cs259c/syllabus.html}}. As the anonymous tunnels, our core function of the application, are dependent on both Dispersy and Tribler we also need M2Crypto. M2Crypto itself is dependent on Python and OpenSSL as it is a Python package using the OpenSSL implementation.
		
			\item PyCrypto\\
			PyCrypto is a Python library which implements certain cryptography functions used by the Tribler package. PyCrypto itself depends on functionality of the OpenSSL package.
		
			\item Boost\footnote{\href{http://www.boost.org/}{www.boost.org}}\\
			The libtorrent library is used to download torrent files with the Tor protocol. To compile libtorrent, the Boost package is required. Boost is a C++\footnote{\href{http://isocpp.org/}{www.isocpp.org}} library which provides code for multithreading, regex, math and asynchronous operations. We have to compile the Boost library with Python bindings enabled so we can invoke the library from Python code.
		
			\item netifaces\\
			The netifaces package provides methods for resolving addresses in a network. It is dependent on Python and used by the Dispersy package.
		
			\item Zope\footnote{\href{http://www.zope.org/}{www.zope.org}}\\
			Zope is an open source web framework for object-oriented web application servers. The Twisted package makes use of this framework for asynchronous networking tasks.
		
			\item Twisted\footnote{\href{https://twistedmatrix.com/trac/}{www.twistedmatrix.com/trac/}}\\
			Twisted is an extensible framework for asynchronous networking written in Python. The framework has special focus on event-based network programming and multi-protocol integration. It has dependencies on the Zope framework. In the Tribler software, Twisted is used for callbacks of network events.
		
			\item anontunnels\\
			This package is the core of our application and contains the code needed for the anonymous tunnels. This package actually contains another package: the support for the Socks5 proxies. The files for this package come from pull request 525 on the Tribler GitHub page.
			
			We made some minor changes to this code: we have changed the master key of the anontunnel community to create our own testing environment. Several paths have been changed to ensure the anontunnels run on Android.
			
			\item Tribler\\
			This package has been obtained from the Tribler devel branch\footnote{\href{https://github.com/Tribler/tribler}{www.github.com/Tribler/tribler}}. Changes to the source code were necessary to run this package on Android. In later versions these changes were reverted and the only modifications were to the Tribler path variables.
			
			\item Dispersy\\
			Since the code of the anonymous tunnels is using Dispersy for node discovery and data synchronization, we have created a Python package with all the code that is needed for Dispersy. This package can be run standalone.\\
			The files we have bundled are from pull request 525 on the Tribler GitHub. We did not use the files from the official Dispersy GitHub because this build was missing some classes we needed (for example, the \texttt{decorator.py}). Besides that, some changes in the pull request have been made to the Dispersy core to add support for the anonymous tunnels.
		
		\end{itemize}
	
	\section{Porting the anonymous tunnels to Android}
		One of our first challenges was to port the code that sets up the anonymous communication to the Android device, using Python for Android. We started by inspecting the current code from pull request 525 and dived into the dependencies this code has with the Tribler core and Dispersy. We decided to create three packages: one package with the Dispersy code, one package with the files we needed from the Tribler core and one package containing the anonymous tunnels code.
		
		After we created these packages, we had to find out which other packages we needed to run everything. To do this, we imported the dependencies in our application and ran it on the device. Each test run, we examined the import error and added the missing dependency. We used various packages that The Global Square has ported such as M2Crypto and netifaces.
		
		Import errors were not the only issue we ran into: we had some problems with the netifaces package. Since this package is copied into the final APK file as a Python egg\footnote{\href{http://mrtopf.de/blog/en/a-small-introduction-to-python-eggs/}{www.mrtopf.de/blog/en/a-small-introduction-to-python-eggs}}, it will be extracted on the device. This failed because the application did not have writing permissions. To solve this, we specified the egg extraction path and made sure the application has writing permissions to that path.
		
		We also found out that some files were missing and not copied into the APK. The \texttt{curves.ec} file was missing and is needed by the cryptography classes found in the Tribler core. We also needed the configuration file of the logger, \texttt{logger.conf}. To make sure these files are part of the application, we copied them into the final application in our build script.
		
	\section{Attempt to compile libtorrent for Android}
		To get libtorrent working on Android there are several big obstacles:
		\begin{itemize}
			\item Compiling Boost for the ARM architecture.
			\item Compiling libtorrent for the ARM architecture.
			\item Compiling libtorrent Python bindings for the ARM architecture.
		\end{itemize}
		
		The official documentation of libtorrent states multiple ways to build libtorrent. The first way is using Boost's build system \emph{Jam}. The second way is using \emph{automake}\footnote{\href{http://www.gnu.org/software/automake/}{www.gnu.org/software/automake/}}. However, before we can compile the libtorrent source code with either of these methods, several modifications had to be applied.
		
		\subsection{Source code modifications}
				
		We found several modifications to the source code to be necessary to compile libtorrent. These modifications are described below.
		
		\begin{itemize}
		\item \texttt{INT64\_MAX} is not defined for Android, so we have to specifically define it.
		\item Multiple environments are defined in \texttt{include/libtorrent/config.hpp}. We add an environment for \texttt{ANDROID}, which sets the following options:
			\begin{itemize}
				\item \texttt{FALLOCATE} is disabled
				\item \texttt{ICONV} is disabled
				\item \texttt{IFADDRS} is disabled
				\item \texttt{MEMALIGN} is enabled
			\end{itemize}
		\item Instead of \texttt{<sys/statvfs.h>} we include \texttt{<sys/vfs.h>}, and redefine \texttt{statvfs} and \texttt{fstatvfs}. This is necessary because the Android libraries only have \texttt{sys/vfs.h} and not \texttt{sys/statvfs}.
		\item Finally, we add an include for \texttt{<sys/syscall.h>} and redefine \texttt{lseek} to \texttt{lseek64}.
		\end{itemize}
		
		\subsection{Boost Jam}
		Boost Jam\footnote{\href{http://www.boost.org/boost-build2/doc/html/bbv2/jam.html}{www.boost.org/boost-build2/doc/html/bbv2/jam.html}} is a build environment created specifically for Boost. However, it can be used to build other software. It is often included as a build option for software that is dependent on Boost, such as libtorrent.
		
		Running Boost Jam is straight forward. After executing \texttt{bootstrap.sh} we can run \texttt{b2} or \texttt{bjam} to compile Boost itself. With this command we specify the architecture which is described in more detail in the \texttt{user-config.jam} configuration file. The resulting compiled library files are compatible with the ARM architecture.
		
		Compiling libtorrent using this method is more advanced. We again specify a \texttt{user-config.jam} with appropriate settings. However, the build process fails during compilation. The source for libtorrent will have to be modified in several places, because the ARM compiler and libraries differ from the normal GNU\footnote{\href{http://gcc.gnu.org}{gcc.gnu.org}} compiler. After modifying the source code, the build still fails. Linker errors occur when we try to compile the Python bindings. The standard Unix\footnote{\href{http://www.unix.org}{www.unix.org}} libraries \texttt{pthread} and \texttt{util} do not have to be linked on Android, yet the Boost Jam environment forces these options for the Python bindings compilation. Due to the complexity of Boost Jam build environment, we decided to try using automake.
		
		\subsection{Automake}
		Automake is a standard set of tools for Unix-based systems that makes it more convenient to configure and compile software on a wide variety of systems. The tools are designed in such a way that it is possible to configure the build process using a simple script.
		
		Following libtorrent's official documentation\footnote{\href{http://libtorrent.org/manual.html}{www.libtorrent.org/manual.html}} we first run \texttt{bootstrap.sh}. Now, in order to configure and compile we will have to set up an environment in which it will use the Android GNU ARM compiler. To do this, we set the following environment variables:
		
		\begin{lstlisting}
export SYSROOT=$ANDROIDNDK/platforms/android-14/arch-arm
export PATH=/usr/local/gcc-4.8.0-arm-linux-androideabi/bin:$PATH
export CC=arm-linux-androideabi-gcc
export CXX=arm-linux-androideabi-g++
export CROSSHOST=arm-linux-androideabi
export CROSSHOME=/usr/local/gcc-4.8.0-arm-linux-androideabi
		\end{lstlisting}
		
		Note that we have set up a custom NDK toolchain. More information about setting up a custom toolchain can be found in Appendix \ref{cpt:libtorrent}.
		
		Compiling libtorrent with this set-up works, but the Python bindings still gives linker errors. These are the same errors as Boost Jam is showing. The linker tries to link \texttt{pthread} and \texttt{util}, which are not required on Android.
		
		We will move two items involving libtorrent to the next scrum iterations:
		\begin{itemize}
		\item Compiling Python bindings for libtorrent
		\item Creating a proof-of-concept application to test if libtorrent works natively and with Python bindings.
		\end{itemize}
	
	\section{Creating a Graphical User Interface with Kivy}
		The first version of AT3 used the standard output for printing status information. This required the phone to be connected to a computer so we can examine the log with the ADB\footnote{\href{http://developer.android.com/tools/help/adb.html}{developer.android.com/tools/help/adb.html}} logcat tool. That is why we decided to create a Graphical User Interface (GUI) for our application. The purpose of this application is to provide a button to start the tunneling and a log to display the status of the application.
	
		Creating a GUI was a small step for us: we already included the Kivy package in our Python for Android distribution. Creating interfaces in Kivy is similar to creating user interfaces for Android with Java: the layout is specified in Kivy files which have the \emph{.kv} extension. In Python this interface file is loaded.
	
	\section{Sprint evaluation}
		During this sprint, we did not manage to complete all the goals set for this sprint. We did not succeed in compiling the libtorrent library. However, because we worked in parallel, we did start with some tasks we had set for the next sprint.
		
		We have talked with Jaap van Touw, a member of the Tribler team. He told us that in his 20 weeks of work, he never managed to get the latest libtorrent to compile for Android.
		Currently he runs an old and modified libtorrent version. We managed to get in contact with Steeve Morin. He got the latest version of libtorrent working with Go bindings on Android. He gave us advice on compiling libtorrent on Android, possibly with Python bindings. We have set the libtorrent package as a separate goal for the next sprint.
