\chapter{Scrum iteration 2: unit testing, relaying and libtorrent}
\label{cpt:iteration2}
	In this chapter we describe our second sprint that lasted two weeks.

	\section{Goals}
		This sprint, we had the following goals:
	
		\begin{enumerate}
			\item Get libtorrent from the first sprint working (must have, libtorrent is a high priority and we need libtorrent for our further work).
			\item Set up a testing environment on Jenkins (should have, applications must be tested thoroughly to ensure everything is in order).
			\item Get the anontunnels running, relaying and downloading (should have, we should investigate how these tunnels are working and how we can trigger a download, even if libtorrent is not working yet).
			\item Merge our work with the other group to create a first prototype of the complete application (would have, this is very dependent on the progress of the other group).
		\end{enumerate}
	
	\section{Jenkins}
		\label{sec:jenkins}
		In this sprint we decided to make use of the continuous integration system called Jenkins\footnote{\url{http://jenkins-ci.org}}, which is already in use by the Tribler development team. Jenkins automatically runs specified tests when a build has been changed or a pull request comes in / has changed. This is a good addition to improve and maintain the (code) quality of our application.
		
		Jenkins provides an environment to run tests in, which is perfect for our setup. Currently Jenkins executes the following steps:
		
		\begin{enumerate}
			\item Jenkins cleans the environment when a build starts, so previous test runs do not influence the outcome of the test.
			\item It then clones our repository and the Python for Android framework from GitHub.
			\item If one of the tests fail, Jenkins will mark the build as failed. Otherwise the build succeeds.
			\item Finally, once the tests are done, the Tribler IRC\footnote{\url{http://tools.ietf.org/html/rfc1459.html}} bot (an automated program that can insert message into an IRC chat group) reports the results of the test in our IRC chat group.
		\end{enumerate} 
		
		Because this is done for every change and pull request, we can closely monitor if changes have unexpected side effects. If they do, we can address them immediately to prevent the problem from spreading or growing more complex if the number of dependencies increase.
		
	
	\section{Libtorrent}
		We have continued working on libtorrent in this sprint and have made major progress in getting it to work on Android devices. Using Steeve's help from last sprint we were able to get a basic version of libtorrent to compile and link for Android. In this sprint we wanted to test the compiled version and build Python bindings.
		
		\subsection{Testing libtorrent with a simple application}
			We have built a simple test application with JNI\footnote{\url{http://docs.oracle.com/javase/7/docs/technotes/guides/jni/}} C++ bindings in order to test whether libtorrent actually works. This application does the minimal work required for downloading a torrent:
			\begin{itemize}
				\item It sets up a session.
				\item It starts listening.
				\item It opens a torrent file.
				\item It keeps looping while requesting status updates to keep track of the progress.
			\end{itemize}
			This very simple torrent client turned out to work well on a Galaxy S2 device. We were able to download the official Ubuntu\footnote{\url{http://www.ubuntu.com}} 14.04 distribution without problems. By manually checking the MD5\footnote{\url{http://www.ietf.org/rfc/rfc1321.txt}} checksum of the file we were able to verify that it was downloaded successfully. We have chosen to download this particular torrent because it has many seeders and is big enough for a good test run.
		
		\subsection{Compiling python bindings}
			To compile Python bindings, one has to add the \texttt{--enable-python-binding} to the \texttt{configure} call. However, doing that in our case causes the configure process to fail. The gcc-arm toolchain can not link with \texttt{-lpthread} and \texttt{-lutil}.
			
			After more investigation we have found out what causes this. Essentially the configure process calls a bunch of \emph{.m4}\footnote{\url{https://www.gnu.org/software/m4/m4.html}} files which try to find out the Python compilation settings automatically. This is done by running a Python interpreter and printing specific values obtained from distutils\footnote{\url{https://docs.python.org/2/library/distutils.html}}. However, it turns out that the python process that gets run is the system Python installation (from Ubuntu). This does not match the same settings of Python for Androids its interpreter. To solve this we had to set some environment variables to point to the Python for Android interpreter:
			
			\begin{itemize}
				\item \texttt{PYTHON = /path/to/python-for-android/build/python/Python2.7.2/hostpython}
				\item \texttt{PYTHON\_CPP\_FLAGS="-I/path/to/python-for-android/python/Python2.7.2/Include"}
			\end{itemize}
			
			After setting this, the configure and compilation process runs fine and is able to create a \emph{libtorrent.so} file with Python bindings. The compiled library works on a Galaxy S2 device. We created a simple Python application that downloads the Ubuntu distribution. This application runs without problems on a Samsung Galaxy S2.
			
		\subsection{Segmentation faults on other devices}
			Running libtorrent, either the native JNI/C++ or the Python bindings version results in segmentation faults on some devices. For example, the Sony Xperia Z throws segmentation faults frequently, whereas this never occurs on the Samsung Galaxy S2.  After a lot of debugging it is still not exactly clear what triggers these segmentation faults.
		
	\section{Downloading over the anonymous tunnels}
	One of the goals of this sprint was to find out how we can trigger a download of a torrent. We started to do this shortly after the libtorrent library worked on Android.
	
	We found that modifications to our existing code were necessary because we discovered that a Tribler session is required to start the download of torrent files. The download itself is managed in the LibtorrentMgr class which is part of the Tribler package. The Tribler session is initialized when starting up Tribler with the GUI. This session also initializes Dispersy and handles the loading of the proxy community (among other communities). Once everything is running we download a file and verify its contents.
	
	Our next step was to port this code to Android. After several import errors (for example, we had to remove the ncurses import and use apsw for the database transactions), they started to run. We first tried to download a torrent file with the computer as proxy. In some cases, this download triggers a segmentation fault which means that something goes wrong with regards to the libtorrent library (or the Python bindings). The error does not always show up: most of the time it occurred during the download. The application crashes and the download is stopped.
	
	This is a serious issue we should further look into. Since the occurrence of the segmentation fault is random, it is hard to debug. We had several attempts to trace down the error, by disabling the anonymous download and using the Tribler session with minimal settings but our application still crashes during the download process.

	\section{Shell script unit tests}
		As we make use of shell scripts to set up variables, run checks and build the application with, we created unit tests.
		
		The first thing we had to do is look for a shell script test framework. As no official test framework is available, we had to search for one that suits our needs. After comparing some frameworks, we decided to go with the shUnit2\footnote{\url{http://shunit.sourceforge.net}} test framework. This framework is lightweight and has some of the standard testing functions such as \emph{AssertEqual}, \emph{AssertTrue} and \emph{AssertFalse} which are all we need. 
		
		In total we have created fourteen tests that checks the following points:
		
		\begin{itemize}
			\item The required export variables.
			\item Whether certain necessary files exist.
			\item Build the application and test if everything runs correctly.
		\end{itemize}
		
		These tests cover all functions present in our build script and all possible subroutines. All of these tests are integrated in Jenkins as described in Section \ref{sec:jenkins}. This means whenever a pull request comes in that modifies code or packages that the application is dependent on, it will run the test to ensure the application still can be built and does not throw errors while building.

	\section{Sprint evaluation}
		This sprint was a setback for us. Even though libtorrent throws a segmentation fault from time to time, we still managed to complete some download runs. Now that we have Jenkins up and running, we can focus on writing more automatic tests to measure changes and maintain a working prototype. The shell script tests we wrote during this sprint should provide a good way to ensure that our builds are correctly configured.
		
		For the next sprint, we will look into the issue of the segmentation faults. As for Jenkins, we will write unit tests to test our written Python code and using the Kivy recorder to apply application tests on our application. These tests will also be sent to the Software Improvement Group (SIG) for evaluation.
