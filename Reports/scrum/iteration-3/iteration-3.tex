\chapter{Scrum iteration 3: stabilizing libtorrent and experiments}
\label{cpt:iteration3}
	In this chapter we describe our third sprint that lasted two weeks.

	\section{Goals}
		During this sprint, we had the following goals:
	
		\begin{enumerate}
			\item Write unit tests for our Python code and application tests for our application (must have: applications must be tested thoroughly to ensure all possible settings are working correctly). 
			\item Send our tests to SIG (must have: this is required for the bachelor project).
			\item Investigate the segmentation fault and check for alternative solutions (should have: libtorrent is close to being stable, but due to time constraints we give priority to writing tests for SIG).
			\item Update our application to use the new Tribler code (should have: while it is not a critical feature, it is good practice to keep the code up to date).
			\item Measure CPU usage and download rates of our application (should have: while it is not needed in order to run the application, it is good to have some performance measurements).
		\end{enumerate}
		
	\section{Application tests}
		In order to verify that the application is working correctly we have our tests running on Jenkins. Using the default unit test framework provided by Python and the Kivy recorder we were able to setup basic user interface tests. These tests run the application, perform a series of user actions such as clicking buttons, and finally assert the state of the application. The tests currently run successfully on Jenkins, which means that future additions to the code will automatically be tested.
		
	\section{RUTracker libtorrent}
		Since we would like to have a stable libtorrent library that downloads a torrent without segmentation faults, we decided to try out the libtorrent that Jaap van Touw used in his bachelor project. This is an older version of libtorrent but has proven to be successful in his project. Jaap van Touw provided us with a link to a GitHub page that contains instructions on how to build libtorrent from source\footnote{\href{https://github.com/javto/tribler-streaming}{www.github.com/javto/tribler-streaming}}. This libtorrent version is used in an Android application called RUTracker\footnote{\href{http://softwarrior.googlecode.com/svn/tags/RutrackerDownloader/2.6.5.5/}{softwarrior.googlecode.com/svn/tags/RutrackerDownloader/2.6.5.5/}}, an Android application that allows users to download torrent files in the background. We are confident that this version of libtorrent is stable.
		
		Instead of using a custom toolchain like we did to compile libtorrent-rasterbar, we used the toolchain that ships with the NDK. The first step was to build some Boost libraries libtorrent depends on (\emph{Boost.filesystem}, \emph{Boost.system} and \emph{Boost.thread}). We used the Boost for Android project\footnote{\href{https://github.com/MysticTreeGames/Boost-for-Android}{www.github.com/MysticTreeGames/Boost-for-Android}} to build Boost 1.49. After that, we compiled libtorrent according to the instructions that Jaap van Touw provided for us. We linked against the Boost libraries we just compiled and we got a library file that we can use in our Android application.
		
		\subsection{Python bindings}
			After writing a small example to test the stability of libtorrent, we came to the conclusion that it does not crash. Since this libtorrent version looked promising, we delved into the Python bindings that we need to communicate between Python and a native C library. Importing this library without Python bindings, results in an error that an initializer function could not be found.
		
			For the Python bindings, it is convenient to use the \emph{Boost.python} library. This library contains several macros and methods to easily define our Python calls. The macro \newline BOOST\_PYTHON\_MODULE, declared in \emph{boost/python.hpp}, initializes our Python library and makes it ready for an import in Python. However, when running a minimal Python script that only imports libtorrent, a segmentation fault is thrown. This means that we are unable to use this version of libtorrent in Python. We are not sure what the cause of this error is. If the initialization and import of the library would work correctly, we could write our own bindings for the libtorrent functions and methods that we need in Tribler.
			
	\section{Libtorrent RC2 progress}
		While we were working on the Russian libtorrent version, we also kept working on the second Release Candidate (RC2\footnote{\href{http://sourceforge.net/projects/libtorrent/files/libtorrent/}{www.sourceforge.net/projects/libtorrent/files/libtorrent/}}) of libtorrent we originally tried. The first step we took, was to compile libtorrent with asserts on. We did this to gain a better understanding in why the segmentation fault occurred. An assert had indeed triggered: in the destructor of the Torrent class, the m\_abort variable should be true but it was false. This could mean that the Torrent object is released too soon.
		
		We decided to post the issue to the official libtorrent bug tracker\footnote{\href{https://code.google.com/p/libtorrent/issues/detail?id=627}{code.google.com/p/libtorrent/issues/detail?id=627}} where we got in contact with a libtorrent developer named Arvid Norberg. He helped us fix the error and gave us the advice to compile with the \texttt{BOOST\_SP\_USE\_PTHREADS} flag. When we tried to compile with this flag we still got the segmentation fault. We took a closer look at what this flag exactly does. It turns out that this flag is responsible for the shared pointers: according to Arvid Norberg, with this define, the shared pointers are using mutex operations instead of atomic operations. He had some bad experience with atomic operations on embedded devices. Using this flag we were able to compile a stable version of libtorrent.
				
	\section{Updating the Tribler package}
		One of the goals of this sprint was to update the Tribler package we are using. The Tribler repository was updated. Twisted was updated to version 14.0.0, certain callbacks were removed and a Twisted Reactor\footnote{\href{https://twistedmatrix.com/documents/12.0.0/core/howto/reactor-basics.html}{twistedmatrix.com/documents/12.0.0/core/howto/reactor-basics.html}} was introduced.
		When we began updating, we discovered that the default recipe in the Python for Android framework required adaptation, upgrading the version number did not work. After inspecting the recipe we asked for advice on the Kivy repository, and they provided us with an updated recipe for Twisted 13.1.0. The updated recipe was also compatible with the latest Twisted release.
		
		The next step was updating the Tribler package. Previously, we downloaded the package and adapted it. Because it was not forked on GitHub we could not automatically update it. We decided to fix this issue immediately and thus created a fork of the main Tribler branch. From this fork we modified the Tribler path variables to make the new Tribler code compatible with our application. This is necessary because several files that get opened by Tribler assume the working directory is the Tribler root. This does not work when executing on more exotic environments such as Android, where the working directory might be different.
		
		Since our package is now a fork, future updates are more easy to merge into our package using the GitHub merge functionality.
			
	\section{Relaying and downloading over multiple hops/multiple circuits}
		In our last sprint, we had some issues with downloading and relaying over multiple hops / multiple circuits. Downloading the 50 MB test file was working correctly over one circuit with one hop, however, adjusting the values of the minimum amount of circuits required for the downloading and the length of the hop would not work.
		
		To debug this issue, we got in contact with the original authors of the anontunnels code. They suggested to place loggers when receiving messages from other nodes and look closely to the incoming messages while disabling the encryption. We logged the messages but we could not find anything that could cause the application to reject circuits. The same code, when executed on a laptop, can successfully download over multiple circuits with multiple hops.
		
		At this time, we were quite some commits behind the devel branch of Tribler. After updating and merging our package with the newest code, we tried again and the downloads were starting. We immediately tried to download the anonymous test file over four circuits and three hops and it worked fine. Also other lengths and other amounts of circuits were working correctly. During the tests (see Chapter \ref{cpt:experiments}) we came to the conclusion that three hop encryption is too heavy for a smartphone.
		
		In order to verify the stability of relaying and downloading, even with encryption disabled, we decided to write a test application that can keep running overnight. We can start some anontunnel instances on our own computers and let the test application download the torrent. After the download is finished, the application restarts and the process starts over. The output is redirected to a log file so we can see issues or problems that occurred during the download.
			
	\section{Sprint evaluation}
		We managed to stabilize libtorrent, updated our Tribler package and made preparations to apply future updates more easy. Furthermore we managed to identify the problem with our download and relaying issue with multiple hops and multiple circuits. The stable environment and the option to download via multiple hops and circuits allowed us to gather data and generate graphs.
		
		The additional unit tests contributed to the stability and robustness of our application. These new tests along with the previous ones from iteration 2 were sent to SIG for evaluation.
		
		Looking back we can conclude that this sprint was successful. The most difficult challenges have been overcome.
