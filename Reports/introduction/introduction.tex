% begrippen die nog niet terug zijn gekomen, referencen of uitleggen
% tor is een protocol
% ref naar dispersy
% linkje naar de afdeling
% uitleggen waarom apps vaak miljarden gebruikers hebben
% beter uitleggen wat het verschil is tussen de twee bachelor eindproject groepen -> in problem definition

\chapter{Introduction}
In the last years, censorship is becoming more and more apparent. In countries like China and North Korea, the freedom of speech is at issue and the Internet is censored. Other countries such as Syria or Ukraine, which are currently in a state of turmoil, have their communications strictly monitored by both military and government. Freedom of speech is practically non-existent in those countries. To make sure people are still able to freely communicate, anonymity is of paramount importance. A popular network for anonymous communication is Tor. However, Tor has disadvantages which limits it's possibilities. More about this, can be read in the paper of Dingledine et al. \cite{dingledine2009performance}

Tribler is a fully decentralized peer-to-peer file sharing system developed by the Parallel and Distributed Systems group\footnote{http://www.pds.ewi.tudelft.nl/} at Delft University of Technology\footnote{http://www.tudelft.nl/en/}. The Tribler development team\footnote{http://www.tribler.org/trac} has recently introduced the anontunnels\footnote{https://github.com/Tribler/tribler/wiki/Anonymous-Downloading-and-Streaming-specifications}, a Tor replacement in a single Dispersy\cite{zeilemaker2013dispersy} community. The decentralized nature of the anontunnels makes it an alternative to Tor for anonymous file sharing.

Mobile phones are increasingly being used for browsing the Internet and streaming video. Smartphones have surpassed 1 billion active users and are expected to double that amount by 2015 \cite{yang2015smartphones}. Popular mobile applications like WhatsApp have billions of users \cite{googleplayinstagram, googleplaywhatsapp}. This shows that Tribler could gain popularity in this area. Smartphones serve as an excellent platform for video recording and sharing due to their mobility and connectivity.

In this project, together with another bachelor project, we work towards achieving anonymous video streaming on Android\footnote{http://www.android.com} smartphones. Our goals are to get the core Tribler peer-to-peer file sharing and the anontunnels working on Android devices. Since Tribler is written in the Python programming language, it is very important to get Python working on Android. Additionally, we want to get all Tribler dependencies such as libtorrent and Dispersy -- which are they key components for downloading -- working on Android.

This thesis describes the prototype we have built and explains the steps we have taken to analyze our prototype. In Chapter \ref{cpt:problemdefinition}, we give the problem definition we want to investigate. In Chapter \ref{cpt:priorwork}, we explain the prior work that has been done, regarding our project. In Chapter \ref{cpt:softwarearchitecture}, we elaborate our software architecture. Chapters \ref{cpt:iteration1}, \ref{cpt:iteration2} and \ref{cpt:iteration3} show our scrum iteration backlogs and explain in each scrum sprint what we have achieved and what not. Followed by Chapter \ref{cpt:experiments}, we describe the experiments we have conducted with our prototype during one of the sprints. In Chapter \ref{cpt:contributions} we discuss our contributions to the open source community and explain what other people can use from our work. Finally, we end this thesis with our conclusions in Chapter \ref{cpt:conclusion}. We recommend the reader to first read our plan of action which can be found in Appendix \ref{cpt:planofaction}.
