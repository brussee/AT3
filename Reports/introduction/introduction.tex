\chapter{Introduction}
In recent years, censorship has become more and more apparent. In countries like China and North Korea, freedom of speech is at issue and the Internet is censored. Other countries such as Syria or Ukraine, which are currently in a state of turmoil, have their internet communications strictly monitored by both the military and the government. Freedom of speech is practically non-existent in those countries. To make sure people are still able to freely communicate, anonymity is of paramount importance. A popular protocol for anonymous internet communication is Tor\footnote{\href{https://www.torproject.org/}{www.torproject.org/}}. However, Tor has disadvantages which limits its possibilities. More about this, can be read in the paper of Dingledine et al. \cite{dingledine2009performance}

Tribler is a fully decentralized peer-to-peer file sharing system developed by the Parallel and Distributed Systems group\footnote{\href{http://www.pds.ewi.tudelft.nl/}{www.pds.ewi.tudelft.nl/}} at Delft University of Technology\footnote{\href{http://www.tudelft.nl/en/}{www.tudelft.nl/en/}}. The Tribler development team\footnote{\href{http://www.tribler.org/trac}{www.tribler.org/trac}} has recently introduced the anontunnels\footnote{\href{https://github.com/Tribler/tribler/wiki/Anonymous-Downloading-and-Streaming-specifications}{www.github.com/Tribler/tribler/wiki/Anonymous-Downloading-and-Streaming-specifications}}, a Tor replacement in a single Dispersy\cite{zeilemaker2013dispersy} community. The decentralized nature of the anontunnels makes it an interesting alternative to Tor for anonymous file sharing.

Mobile phones are increasingly being used for browsing the Internet and streaming video. Smartphones have surpassed 1 billion active users and are expected to double that amount by 2015 \cite{yang2015smartphones}. Popular mobile applications like WhatsApp have billions of users \cite{googleplayinstagram, googleplaywhatsapp}. Smartphones serve as an excellent platform for video recording and sharing due to their mobility and connectivity. Tribler's usage and popularity can be increased by supporting the smartphone operating system Android\footnote{\href{http://www.android.com/about/}{www.android.com/about/}}.

In this project we work towards achieving anonymous video streaming on Android smartphones. Our goals are to get Tribler and the anontunnels working on Android devices. Since Tribler is written in the Python programming language, it is very important to get Python working on Android. Additionally, we want to get all Tribler dependencies such as libtorrent and Dispersy -- which are the key components for downloading -- working on Android.

This thesis describes the prototype we have built and explains the steps we have taken to analyze our prototype. In Chapter \ref{cpt:problemdefinition}, a problem definition is provided and we motivate the research question. In Chapter \ref{cpt:priorwork}, we explain the prior work that has been done, regarding our project. In Chapter \ref{cpt:softwarearchitecture}, we elaborate our software architecture. Chapters \ref{cpt:iteration1}, \ref{cpt:iteration2} and \ref{cpt:iteration3} show our scrum iteration reports and explain in each scrum sprint what we have achieved and what not. Chapter \ref{cpt:experiments} presents the experiments we have conducted with our prototype. In Chapter \ref{cpt:contributions} we sum up our contributions to the open source community and explain what other people can use from our work. Finally, we conclude this thesis in Chapter \ref{cpt:conclusion}. We recommend the reader to first read our plan of action which can be found in Appendix \ref{cpt:planofaction}.
