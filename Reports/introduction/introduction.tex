\chapter{Introduction}

In the last years, censorship is becoming more and more apparent. In countries like China and North Korea, the freedom of speech is at issue and the Internet is censored. Inhabitants are prohibited from visiting websites that are forbidden by the government. Other countries such as Syria or Ukraine, which are currently in a state of turmoil, have their communications strictly monitored by both military and government. Freedom of speech is practically non existent in those countries. To make sure people are still able to freely communicate, anonymity is of paramount importance. A popular network for anonymous communication is Tor. However, Tor suffers from several problems and does not scale very well.

Tribler is a fully decentralized peer-to-peer file sharing system developed by the Parallel and Distributed Systems group at Delft University of Technology. The Tribler development team has recently introduced the anontunnels, a Tor-like protocol over UDP. The decentralized nature of the anontunnels makes it an alternative to Tor for anonymous file sharing.

Mobile phones are increasingly being used for browsing the internet and streaming video. Smartphones have surpassed 1 billion active users and are expected to double that amount by 2015 \cite{yang2015smartphones}. Popular mobile applications like WhatsApp have billions of users \cite{googleplayinstagram, googleplaywhatsapp}. Smartphones serve as an excellent platform for video recording and sharing due to their mobility and connectivity.

In this project, together with another bachelor project, we work towards achieving anonymous video streaming on Android smartphones. Our goals are to get the core Tribler peer-to-peer filesharing and the anontunnels working on Android devices. Since Tribler is written in the python programming language, it is very important to get python working on Android. Additionally, we want to get all Tribler dependencies such as libtorrent and Dispersy working on Android.