\section{Motivation}
	Worldwide the amount of mobile devices is growing fast. In 2013, there were 6.8 billion mobile subscriptions worldwide \cite{itustatistics}. This number is expected to exceed the world population in 2014 with 7.3 billion subscriptions. Android had a share of 81\% in Q3 of devices shipped \cite{forbesandroidmarket}, this shows that there is a lot of potential gain on the Android market. 
	
	Android is open source and can be modified more than a platform like iOS, on top of that, applications submitted to the Google Play are not subject to a review before you can release them, which allows for an easier deployment. Since the Tor-tunnel functionality is written in Python and a library called Python for Android already exists, it is an even more natural choice to develop it for Android.

	Besides the arguments stated above, a mobile application is an excellent way to attract more new users to Tribler. While the Tor project does have an Android application named Orbot, this application does not provide anonymous file transfer \cite{tororbot, googleplayorbot}. Orbot only serves as an anonymous proxy. Also, the structure of Tor is currently semi-centralized \cite{jagerman2014fifteen} where this application will be completely decentralized using Dispersy as synchronization system \cite{zeilemaker2013dispersy}. This means that this application will be less vulnerable to attacks that are targeted against central components of the network, as there are less. Moreover, this application can easily scale where Tor is restricted to 1.2 million nodes \cite{mclachlan2009scalable}. With the billions of mobile devices out there the and no competition in this area, this application has a lot of potential to open a new way of anonymous file sharing while being independent of computers.
	
