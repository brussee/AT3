% Exacte set-up
% Welke computers? Welk Wi-Fi netwerk? Relays only op computers?
% Zelfde router -> load balancing -> Dit heeft effect op de metingen
% 
\section{Set-up}
	\label{sec:experiments:approach}
	To measure the performance of the anontunnels running on an Android device, we decided to make use of the Tribler test torrent. This test downloads a 50 MB test file on a Sony Xperia Z (C6603)  connected to the eduroam Wi-Fi network. Running multiple times, each with a different setting of variables such as amount of hops or circuits, we can measure how these variables impact the performance of the CPU usage and download rates.
	
	We are interested in the impact of the amount of hops and circuits on the performance of the anontunnels. The following four configurations were chosen to measure this:
	
	\begin{enumerate}
		\item Download the test file with 1 hop and 1 circuit.
		\item Download the test file with 1 hop and 3 circuits.
		\item Download the test file with 3 hops and 1 circuit.
		\item Download the test file with 3 hops and 3 circuits.
	\end{enumerate}
	
	CPU usage is measured with the \emph{psutil} Python module\footnote{\href{https://pypi.python.org/pypi/psutil}{pypi.python.org/pypi/psutil}}. During the tests we shut down all other applications running on the smartphone. This minimizes the impact other applications have on the CPU measurements.
	
	Bandwidth is measured using the download status obtained from Tribler. This provides us with information about the test file we are downloading. This includes current speed (in KB/s) and the current progress (in percentage) of the download.
	
	We run 10 stand-alone anontunnels on a MacBook Pro 7.1 running Ubuntu 14.04. This computer is connected to the Internet over an ethernet cable with a speed of 100 Mb/s. 
	
	During our experiments we found that the download would not start when enabling cryptography. CPU usage during these attempts would stay at 100\%. This means that either the implementation of the cryptography is bad or the device's hardware is not powerful enough. For this reason we were forced to run the experiments with cryptography disabled.
	
	%To measure the CPU we make use of the \emph{adb shell top} command. Which provides a list of all current running processes. Per process, the CPU usage in percentage is given. From this list we extract our two processes: the application itself  and the anontunnel service where the anontunnels run on. During the test run, we measure the CPU usage each second and write this to a log file. This file is then parsed using a Python script to generate a graph.
	
	%To measure the download rates, we modified the test file to report the current download speed in the callback which is called every second. These speeds are also written to a log file which gets parsed by the same Python script used for parsing the CPU logs.