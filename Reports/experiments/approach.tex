\section{Our approach}
	\label{sec:approach}
	To measure the performance of our device running the anontunnels, we decided to make use of the Tribler test torrent that was setup earlier. This test downloads a 50 MB test file. Running multiple times, each with a different setting of variables such as amount of hops or circuits, we can measure how these variables impact the performance of the CPU usage and download rates.
	
	We came up with the following five configurations to test both the CPU usage and the download rates:
	
	\begin{enumerate}
		\item Measure the CPU when the application is searching for circuits but does not download anything.
		\item Download the test file with the anontunnels off, directly from the source.
		\item Download the test file with 1 hop and 1 circuit, providing security but not anonymity.
		\item Download the test file with 1 hop and 3 circuits. We expect higher download rates here in comparison to the third run.
		\item Download the test file with 3 hops and 1 circuit, providing both security and anonymity.
		\item Download the test file with 3 hops and 3 circuits, providing both security and anonymity. We expect a higher download rate in comparison to the fifth run.
	\end{enumerate}
	
	To measure the CPU we make use of the \emph{adb shell top} command. Which provides a list of all current running processes. Per process, the CPU usage in percentage is given. From this list we extract our two processes: the application itself  and the anontunnel service where the anontunnels run on. During the test run, we measure the CPU usage each second and write this to a log file. This file is then parsed using a Python script to generate a graph.
	
	To measure the download rates, we modified the test file to report the current download speed in the callback which is called every second. These speeds are also written to a log file which gets parsed by the same Python script used for parsing the CPU logs.