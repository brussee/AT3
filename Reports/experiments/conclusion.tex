\section{Conclusion}
\label{sec:experiments:conclusion}
	We come to the conclusion that streaming video over the anonymous tunnels is possible, as long as cryptography is disabled. The average download speed is capable of providing the necessary bitrate for videos with a 720p resolution. % Increasing the hops indeed cause a negative impact on the download rates as there is a higher change of packet loss or a bottleneck in the circuit.
	
	The unforeseen CPU bound that we encountered means that either the cryptographical implementation is bad or the current generation smartphones are not powerful enough. By compiling and running gmpy\footnote{\href{http://www.gmpy.org/}{www.gmpy.org}}, a fast large number library for Python, we could see a large speedup during the Diffie-Hellman handshake. Due to time constraints we were unable to compile and implement this in the current version of the AT3 application.
	
	It was announced that the new ARMv8 architecture will support AES hardware acceleration \cite{armv8anouncement}. Because the anontunnels use AES encryption, this would increase performance and reduce the computational burden on the CPU. %If we assume that the relay nodes do not inspect the data, anonymous downloading is possible. The anontunnels will be able to run on smartphones with both 3 hops and encryption once more powerful hardware becomes available. It is anounced \cite{armv8anouncement} that the new ARM architecture (armv8) will support AES hardware acceleration which should speed up things greatly. We also talked with Niels, one of the lead developers of the Tribler group. He told us that if we manage to get the \emph{gmpy} python package to work on Android, this will speed up the Diffie-Hellman handshakes as well. Since we do not have the time to compile this at this stage, it is not available in our project.	
	
	Without a significant speedup of the cryptographic functions, our conclusion is that the anontunnels will not be able to run with three hops and encryption enabled.