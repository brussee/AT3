\section{Theoretical analysis}
	In this section, we will give a theoretical analysis of our application. We will discuss the bitrate we need to stream a movie. In the next sections, we will perform measurements on our application and see whether our application can be used to stream a movie without problems on an Android device. We will talk about streaming 240p and 480p movies in the next subsections.
	
	\subsection{240p/480p movies}
		We don't need very high quality movies on smartphone screens. 240p or 480p resolution is appropriate for a smartphone screen, however, the quality of the movie is not very good. We will derive the bitrate we need to stream these resolutions and explain which variables can influence the download rates.
		
		If look at the bitrates of YouTube, 240p has a resolution of 426x240 pixels and 480p has a resolution of 640x360 pixels. The recommended bitrate for 240p is 400 Kbps and for 480 it is 750 Kbps. While our Application is different from YouTube, we can estimate with the values of YouTube the bitrate we should need.
		
		Downloading a torrent file with enough seeders on the Android device will probably reach a download speed of at least 750 Kbps, however, adding anonymity, the download speed will decrease. Since anonymity comes with encryption, the CPU usage will also increase (we first need to decrypt the data in incoming packets). We don't know yet whether anonymity has a great impact on the download speed and we will discuss our measurements of CPU usage and download speed.
		
	\subsection{Download speed variables}
		There are some circumstances which causes the download speed to increase or decrease. When downloading a torrent, one of the most important factors that influences the download speed is the amount of seeders. When we don't have enough seeders, we can't expect high download speeds. Another factor is the strength of each seeder: it is possible that some seeders have their upload rate limited or have a bad connection.
		
		When downloading over anontunnels, an important factor is the amount of circuits. When we download over more circuits, the download should be faster. In general, the length of the circuits negatively impacts the download speed. When circuits are longer, there is more chance that there is a bottleneck in the circuit. The maximum speed of the data over one circuit is equal to the minimum bandwidth of all hops on the circuit.
		
		Another important factor we should take into account, is the possibility of random disconnects of peers during the download. When a seeding peer disconnects, the download speed of the torrent will decrease. When a circuit drops, our application will try to setup a new circuit. During this period, a decrease in download speed should be visible.
		
		Since there are many variables that influences the download speed, we can't expect the download speed rate to be stable. When streaming movies, these changing download rates could be problematic. The movie should buffer to take these variable download speeds in account.