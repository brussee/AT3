\section{Theoretical analysis}
	To make sure a video can be downloaded in a reasonable amount of time or streamed instantly, the application must achieve certain minimum bitrates. We will discuss the bitrate we need to stream a movie. In the following sections, we will discuss whether our application can be used to stream a movie without problems on an Android device based on our measurements. Fist, we will talk about streaming 240p and 480p movies in the next subsections.
	
	\subsection{240p/480p movies}
		Most smartphone screens do not have a high resolution, therefore we do not need very high quality movies on smartphone screens. 240p or 480p resolution is appropriate for a smartphone screen, however, the quality of the movie will not be high on other devices. We will derive the bitrate we need to stream these resolutions and explain which variables can influence the download rates.
		
		If we look at the bitrates of YouTube \cite{googlebitrates}, 240p has a resolution of 426x240 pixels and 480p has a resolution of 640x360 pixels. The recommended bitrate for 240p is 400 Kbps and for 480 it is 750 Kbps. While our Application is different from YouTube, we can estimate with the values of YouTube the bitrate we should need.
		
		Downloading a torrent file with enough seeders on the Android device will probably reach a download speed of at least 750 Kbps, however, adding anonymity, the download speed will decrease. Since the anontunnels come with  encryption and circuits with different lengths, the CPU usage will also increase as the device has to decrypt data. The longer the circuit, the more layers of encryption the device has to decrypt. This increased CPU usage
		can have negative impacts on the bitrates as the device can only decrypt a certain amount of bytes per second.
		
	\subsection{Download speed variables}
		There are some circumstances which causes the download speed to increase or decrease. When downloading a torrent, one of the most important factors that influences the download speed is the amount of seeders. When we do not have healthy seeders or a lack of seeders in general, we usually can not expect high download speeds. Another factor is the strength of each seeder: it is possible that some seeders have their upload rate limited or have a bad connection.
		
		When downloading over anontunnels, an important factor is the amount of circuits. When we download over more circuits, the download should be faster. In general, the length of the circuits negatively impacts the download speed. When circuits are longer, there is more chance that there is a bottleneck in the circuit and as mentioned in the previous subsection, the device has to decrypt more layers. The maximum speed of the data over one circuit is equal to the minimum bandwidth of the maximum of all hops on the circuit.
		
		Another important factor we should take into account, is the possibility of random disconnects of peers during the download. When a seeding peer disconnects, the download speed of the torrent will decrease. When a circuit drops, our application will try to setup a new circuit or connect to an existing one. During this period, a decrease in download speed should be visible.
		
		Since there are many variables that influences the download speed, we cannot expect the download speed rate to be stable. When streaming movies, these changing download rates could be problematic if they are averaging the required speed. The movie should buffer to take these variable download speeds into consideration.