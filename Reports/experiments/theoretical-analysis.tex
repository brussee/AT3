\section{Theoretical analysis}
\label{sec:experiments:theoretical}
	To make sure a video can be downloaded in a reasonable amount of time or streamed instantly, the application must achieve certain minimum bitrates. In the following Subsections, we will discuss the bitrate we need to stream a movie.
	
	\subsection{Required bitrate}
	\label{sec:experiments:theoretical:bitrates}
		Most smartphone screens do not have a high resolution, therefore we do not need to be able to stream very high resolution videos. The additional quality is negligible on such a small screen and therefore a waste of resources. For a smartphone, a resolution of 480p (854x480 pixels) or 720p (1280x720 pixels) is appropriate.
		
		YouTube\footnote{\href{https://www.youtube.com/}{www.youtube.com}}, a major video streaming platform, has published recommendations on the various bitrates for different resolutions\cite{googlebitrates}. The recommended bitrate for 480p is 1000 Kb/s and for 720p it is 2500 Kb/s. These values serve as an indication for the bitrates we will need to achieve. 
		
	\subsection{Factors that impact the download speed}
		There are some factors that cause the download speed to increase or decrease. When downloading a torrent, one of the most important factors that influences the download speed is the amount of seeders. When we do not have enough seeders, we generally can not expect high download speeds. Another factor is the strength of each seeder: it is possible that some seeders have their upload rate limited or have a bad connection.
		
		When downloading over anontunnels, an important factor is the amount of circuits. When we download over more circuits, the download should be faster. In general, the length of the circuits negatively impacts the download speed. When circuits are longer, there is a higher chance that there is a bottleneck in the circuit. The maximum speed of the download over one circuit is as fast as the slowest hop.
		
		Another important factor we should take into account, is the possibility of random disconnects of peers during the download. When a seeding peer disconnects, the download speed of the torrent will decrease. When a circuit drops, our application will try to setup a new circuit or connect to an existing one. During this period, a decrease in download speed will occur.
		
		Since there are many variables that influences the download speed, we cannot expect the speed rate to be stable. When streaming movies, these changing download rates could be problematic. A video should buffer to take these variable download speeds into consideration.